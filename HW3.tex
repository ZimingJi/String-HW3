\documentclass[12pt]{article}
 
\usepackage[margin=1in]{geometry}
\usepackage{amsmath,amsthm,amssymb}
\usepackage{physics}
 
\newcommand{\N}{\mathbb{N}}
\newcommand{\R}{\mathbb{R}}
\newcommand{\Z}{\mathbb{Z}}
\newcommand{\Q}{\mathbb{Q}}
 
\newenvironment{theorem}[2][Theorem]{\begin{trivlist}
\item[\hskip \labelsep {\bfseries #1}\hskip \labelsep {\bfseries #2.}]}{\end{trivlist}}
\newenvironment{lemma}[2][Lemma]{\begin{trivlist}
\item[\hskip \labelsep {\bfseries #1}\hskip \labelsep {\bfseries #2.}]}{\end{trivlist}}
\newenvironment{exercise}[2][Exercise]{\begin{trivlist}
\item[\hskip \labelsep {\bfseries #1}\hskip \labelsep {\bfseries #2.}]}{\end{trivlist}}
\newenvironment{problem}[2][Problem]{\begin{trivlist}
\item[\hskip \labelsep {\bfseries #1}\hskip \labelsep {\bfseries #2.}]}{\end{trivlist}}
\newenvironment{question}[2][Question]{\begin{trivlist}
\item[\hskip \labelsep {\bfseries #1}\hskip \labelsep {\bfseries #2.}]}{\end{trivlist}}
\newenvironment{corollary}[2][Corollary]{\begin{trivlist}
\item[\hskip \labelsep {\bfseries #1}\hskip \labelsep {\bfseries #2.}]}{\end{trivlist}}
 
\begin{document}
 
\title{Homework 3}
\author{Ziming Ji\\ 
PHY 539: Introduction to String Theory}
 
\maketitle
 
\section{Problem 1} 
\begin{paragraph}{a)}
It is not hard to show that under $z\to \frac{a z+b}{c z+d}$, the cross ratio becomes
\begin{equation}
\frac{\left(\frac{a z_1+b}{c z_1+d}-\frac{a z_2+b}{c z_2+d}\right) \left(\frac{a z_3+b}{c z_3+d}-\frac{a z_4+b}{c z_4+d}\right)}{\left(\frac{a z_1+b}{c z_1+d}-\frac{a z_3+b}{c z_3+d}\right) \left(\frac{a z_2+b}{c z_2+d}-\frac{a z_4+b}{c z_4+d}\right)}=\frac{\left(z_1-z_2\right) \left(z_3-z_4\right) (b c-a d)^2}{\left(z_1-z_3\right) \left(z_2-z_4\right) (b c-a d)^2}=\frac{\left(z_1-z_2\right) \left(z_3-z_4\right)}{\left(z_1-z_3\right) \left(z_2-z_4\right)}.
\end{equation}
\end{paragraph}
\begin{paragraph}{b)}
We set the transformation to be $z\to \frac{a z+b}{c z+d}$. Solving the set of equations
\begin{equation}
a z_1+b=0\land a z_2+b=c z_2+d\land c z_3+d=0\land a d-b c=1,
\end{equation}
we have 
\begin{equation}
\begin{aligned}[t]
a\to -\frac{\sqrt{z_3-z_2}}{\sqrt{\left(z_1-z_2\right) \left(z_1-z_3\right)}},b\to \frac{z_1 \sqrt{z_3-z_2}}{\sqrt{\left(z_1-z_2\right) \left(z_1-z_3\right)}},\\
c\to \frac{z_2-z_1}{\sqrt{\left(z_1-z_2\right) \left(z_1-z_3\right)} \sqrt{z_3-z_2}},d\to \frac{\left(z_1-z_2\right) z_3}{\sqrt{\left(z_1-z_2\right) \left(z_1-z_3\right)} \sqrt{z_3-z_2}}
\end{aligned}
\end{equation}
or
\begin{equation}
\begin{aligned}[t]
a\to \frac{\sqrt{z_3-z_2}}{\sqrt{\left(z_1-z_2\right) \left(z_1-z_3\right)}},b\to -\frac{z_1 \sqrt{z_3-z_2}}{\sqrt{\left(z_1-z_2\right) \left(z_1-z_3\right)}},\\
c\to \frac{z_1-z_2}{\sqrt{\left(z_1-z_2\right) \left(z_1-z_3\right)} \sqrt{z_3-z_2}},d\to \frac{\left(z_2-z_1\right) z_3}{\sqrt{\left(z_1-z_2\right) \left(z_1-z_3\right)} \sqrt{z_3-z_2}}.
\end{aligned}
\end{equation}
\end{paragraph}
\section{Problem 2}
We refer to the $b$ $c$ system as a $2D$ field theory with a Grassmann action
\begin{equation}
S=\frac{1}{2\pi } \int d^2 z \,\,b  \bar{\partial} c
\end{equation}
The contraction of $b$ and $c$ field is 
\begin{equation}
b(z)c(\omega)\,-:b(z)c(\omega):=\frac{1}{z-\omega}.
\end{equation}
\begin{paragraph}{a)}
\begin{equation}
\begin{split}
T(z) T(w) & =\left(  : (\partial_z b) c(z): - \lambda \partial_z : b c (z): \right) \left(  : (\partial_w b) c(w): - \lambda \partial_w : b c (w): \right)  \\
& =  : (\partial_z b) c(z):  : (\partial_w b) c(w):  - \lambda \partial_z : b c (z):  : (\partial_w b) c(w):  \\
&~~~- \lambda  : (\partial_z b) c(z):\partial_w : b c (w):  + \lambda^2 \partial_z : b c (z):\partial_w : b c (w): 
\end{split}
\end{equation}
The full contraction(quartic) part is
\begin{equation}
\begin{aligned}[t]
\lambda ^2 \frac{\partial }{\partial z}\frac{\partial }{\partial \omega }\frac{1}{(z-\omega ) (z-\omega )}-\lambda  \frac{\partial }{\partial \omega }\frac{\frac{\partial }{\partial z}\frac{1}{z-\omega }}{z-\omega }-\lambda  \frac{\partial }{\partial z}\frac{\frac{\partial }{\partial \omega }\frac{1}{z-\omega }}{z-\omega }+\frac{\partial }{\partial z}\frac{1}{z-\omega } \frac{\partial }{\partial \omega }\frac{1}{z-\omega }=\frac{-6 (\lambda -1) \lambda -1}{(z-\omega )^4}
\end{aligned}
\end{equation}
The other terms are(omitting normal ordering symbols for simplicity)
\begin{equation}
\begin{aligned}[t]
c(\omega ) \frac{\partial b(z)}{\partial z} \frac{\partial }{\partial \omega }\frac{1}{z-\omega }+c(z) \frac{\partial b(\omega )}{\partial \omega } \frac{\partial }{\partial z}\frac{1}{z-\omega }=-\frac{2 \left(c(\omega ) b'(\omega )\right)}{(z-\omega )^2}+\frac{-c(\omega ) b''(\omega )-b'(\omega ) c'(\omega )}{z-\omega }\\
+O\left((z-\omega )^0\right)\\
-\lambda  \frac{\partial }{\partial z}\left(b(z) c(\omega ) \frac{\partial }{\partial \omega }\frac{1}{z-\omega }+\frac{c(z) \frac{\partial b(\omega )}{\partial \omega }}{z-\omega }\right)=\frac{2 \lambda  b(\omega ) c(\omega )}{(z-\omega )^3}+\frac{4 \lambda  c(\omega ) b'(\omega )}{(z-\omega )^2}\\
+\frac{\lambda  \left(2 c(\omega ) b''(\omega )+2 b'(\omega ) c'(\omega )\right)}{z-\omega }+O\left((z-\omega )^0\right)\\
-\lambda  \frac{\partial }{\partial \omega }\left(b(\omega ) c(z) \frac{\partial }{\partial z}\frac{1}{z-\omega }+\frac{c(\omega ) \frac{\partial b(z)}{\partial z}}{z-\omega }\right)=-\frac{2 (\lambda  b(\omega ) c(\omega ))}{(z-\omega )^3}+\frac{\lambda  \left(2 c(\omega ) b'(\omega )-2 b(\omega ) c'(\omega )\right)}{(z-\omega )^2}\\
+\frac{\lambda  \left(c(\omega ) b''(\omega )-b(\omega ) c''(\omega )\right)}{z-\omega }+O\left((z-\omega )^0\right)\\
\lambda ^2 \frac{\partial }{\partial z}\frac{\partial }{\partial \omega }\left(\frac{b(\omega ) c(z)}{z-\omega }+\frac{b(z) c(\omega )}{z-\omega }\right)=-\frac{4 \left(\lambda ^2 c(\omega ) b'(\omega )\right)}{(z-\omega )^2}+\frac{\lambda ^2 \left(-2 c(\omega ) b''(\omega )-2 b'(\omega ) c'(\omega )\right)}{z-\omega }\\
+O\left((z-\omega )^0\right)
\end{aligned}
\end{equation}
In conclusion, we have
\begin{equation}
\begin{aligned}[t]
T(z)T(\omega)\sim \frac{-6 (\lambda -1) \lambda -1}{(z-\omega )^4}-\frac{2 \left((\lambda -1) (2 \lambda -1) c(\omega ) b'(\omega )+\lambda  b(\omega ) c'(\omega )\right)}{(z-\omega )^2}\\
\frac{((3-2 \lambda ) \lambda -1) c(\omega ) b''(\omega )+(-2 (\lambda -1) \lambda -1) b'(\omega ) c'(\omega )-\lambda  b(\omega ) c''(\omega )}{z-\omega }
\end{aligned}
\end{equation}
???Shouldn't $T(z)T(\omega) \sim \frac{-6 (\lambda -1) \lambda -1}{(z-\omega )^4}+\frac{2T(\omega)}{(z-\omega)^2}+\frac{\partial T(\omega)}{z-\omega}$? I only noticed this contradiction in the last minute but could not make it right. 
\end{paragraph}
\begin{paragraph}{b)}
\begin{equation}
\begin{aligned}[t]
T(z)=\sum\limits_{m=-\infty}^{\infty}\frac{L_m}{z^{m+2}}\,\, ,\qquad L_m=\frac{1}{2\pi i}\oint dz \,\, z^{m+1}T(z)
\end{aligned}
\end{equation}
We can just expand $T(z)$ in terms of $b$ and $c$ modes and pick up the power $-(m+2)$ terms for $m\neq 0$.
\begin{equation}
\begin{aligned}[t]
T(z)&= : (\partial_z b) c(z): - \lambda \partial_z : b c (z):=\sum\limits_l\sum\limits_k :\frac{-(k+\lambda):b_k}{z^{k+\lambda+1}}\frac{c_l:}{z^{l+1-\lambda}}:\\
&\qquad-:\lambda(\frac{-(k+\lambda):b_k}{z^{k+\lambda+1}}\frac{c_l:}{z^{l+1-\lambda}}+\frac{:b_k}{z^{k+\lambda}}\frac{-(l+1-\lambda)c_l:}{z^{l+2-\lambda}}):\\
(\text{setting  }l=m-k)&=\sum\limits_{m}\sum\limits_{k}\frac{:b_k c_{m-k:}}{z^{m+2}}(-(k+\lambda)+\lambda(k+\lambda)+\lambda(m-k+1-\lambda))\\
&=\sum\limits_{m}\sum\limits_{k}\frac{:b_k c_{m-k}:}{z^{m+2}}(\lambda  m-k)
\end{aligned}
\end{equation}
It is easy to read off $L_m$ from above as $L_m=\sum\limits_{k}(\lambda  m-k):b_k c_{m-k}:$. For $m=0$, an appropriate normal ordering constant should be included: $L_0=\sum\limits_{k}  k(c_{-k}b_k+b_{-k}c_k) -k$. 
\end{paragraph}
\begin{paragraph}{c)}
\begin{equation}
\begin{aligned}[t]
[L_m,b_n]=[\sum\limits_{k}(\lambda  m-k):b_k c_{m-k}:,b_n]\\
=((\lambda-1)m-n)[:b_{m+n}c_{-n}:,b_n]=((\lambda-1)m-n)b_{m+n}
\end{aligned}
\end{equation}
\begin{equation}
\begin{aligned}[t]
[L_m,c_n]=[\sum\limits_{k}(\lambda  m-k):b_k c_{m-k}:,c_n]=(\lambda m+n)[:b_{-n}c_{m+n}:,c_n]=(\lambda m+n)c_{m+n}
\end{aligned}
\end{equation}
\end{paragraph}
\begin{paragraph}{d)}
Because $c(z)\ket{1}$ has no pole at $z\to 0$, $c_m\ket{1}$ has to be zero for $m>\lambda-1$. Because $b(z)\ket{1}$ has no pole at $z\to 0$, $b_m\ket{1}$ has to be zero for $m>-\lambda$.
\end{paragraph}
\begin{paragraph}{e)}
\begin{equation}
\begin{aligned}[t]
L_0\ket{1}&=(\sum\limits_{k>0}k(c_{-k}b_k+b_{-k}c_k)+a^g)\ket{1}\\
&=(\sum\limits_1^{[-\lambda]}kc_{-k}b_k+\sum\limits_1^{[\lambda-1]}kb_{-k}c_k+a^g)\ket{1}\\
&=(\{\frac{1+[-\lambda]}{2}[-\lambda],\,\text{if }\lambda\leq -1;\,0 \text{ otherwise }\}\\
&+\{\frac{1+[\lambda-1]}{2}[\lambda-1],\,\text{if }\lambda\geq 2;\,0 \text{ otherwise }\}+a^g)\ket{1}
\end{aligned}
\end{equation}
For $L_1$, exchanging $b$ and $c$ operators does not give extra constant term. So the only non zero terms are in the region $2-\lambda\leq k\leq -\lambda, k\in \mathbb{Z}$ which is an empty set. Thus,
\begin{equation}
\begin{aligned}[t]
L_1\ket{1}&=(\sum\limits_{k}(\lambda- k):b_{k}c_{1-k}:))\ket{1}=0
\end{aligned}
\end{equation}
For $L_{-1}$ the non zero term is $k=-\lambda$. If $\lambda\not\in\mathbb{Z}$, it is empty; if $\lambda\in\mathbb{Z}$, the only term left is still zero. So $L_{-1}\ket{1}=0$.
\end{paragraph}
\begin{paragraph}{f)}
Remember that $[L_0,b_n]=-nb_n$ and $[L_0,c_n]=nc_n$
\end{paragraph}
\end{document}
