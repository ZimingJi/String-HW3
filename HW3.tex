\documentclass[12pt]{article}
 
\usepackage[margin=1in]{geometry}
\usepackage{amsmath,amsthm,amssymb}
 
\newcommand{\N}{\mathbb{N}}
\newcommand{\R}{\mathbb{R}}
\newcommand{\Z}{\mathbb{Z}}
\newcommand{\Q}{\mathbb{Q}}
 
\newenvironment{theorem}[2][Theorem]{\begin{trivlist}
\item[\hskip \labelsep {\bfseries #1}\hskip \labelsep {\bfseries #2.}]}{\end{trivlist}}
\newenvironment{lemma}[2][Lemma]{\begin{trivlist}
\item[\hskip \labelsep {\bfseries #1}\hskip \labelsep {\bfseries #2.}]}{\end{trivlist}}
\newenvironment{exercise}[2][Exercise]{\begin{trivlist}
\item[\hskip \labelsep {\bfseries #1}\hskip \labelsep {\bfseries #2.}]}{\end{trivlist}}
\newenvironment{problem}[2][Problem]{\begin{trivlist}
\item[\hskip \labelsep {\bfseries #1}\hskip \labelsep {\bfseries #2.}]}{\end{trivlist}}
\newenvironment{question}[2][Question]{\begin{trivlist}
\item[\hskip \labelsep {\bfseries #1}\hskip \labelsep {\bfseries #2.}]}{\end{trivlist}}
\newenvironment{corollary}[2][Corollary]{\begin{trivlist}
\item[\hskip \labelsep {\bfseries #1}\hskip \labelsep {\bfseries #2.}]}{\end{trivlist}}
 
\begin{document}
 
\title{Homework 3}
\author{Ziming Ji\\ 
PHY 539: Introduction to String Theory}
 
\maketitle
 
\section{Problem 1} 
\begin{paragraph}{a)}
It is not hard to show that under $z\to \frac{a z+b}{c z+d}$, the cross ratio becomes
\begin{equation}
\frac{\left(\frac{a z_1+b}{c z_1+d}-\frac{a z_2+b}{c z_2+d}\right) \left(\frac{a z_3+b}{c z_3+d}-\frac{a z_4+b}{c z_4+d}\right)}{\left(\frac{a z_1+b}{c z_1+d}-\frac{a z_3+b}{c z_3+d}\right) \left(\frac{a z_2+b}{c z_2+d}-\frac{a z_4+b}{c z_4+d}\right)}=\frac{\left(z_1-z_2\right) \left(z_3-z_4\right) (b c-a d)^2}{\left(z_1-z_3\right) \left(z_2-z_4\right) (b c-a d)^2}=\frac{\left(z_1-z_2\right) \left(z_3-z_4\right)}{\left(z_1-z_3\right) \left(z_2-z_4\right)}.
\end{equation}
\end{paragraph}
\begin{paragraph}{b)}
We set the transformation to be $z\to \frac{a z+b}{c z+d}$. Solving the set of equations
\begin{equation}
a z_1+b=0\land a z_2+b=c z_2+d\land c z_3+d=0\land a d-b c=1,
\end{equation}
we have 
\begin{equation}
\begin{aligned}[t]
a\to -\frac{\sqrt{z_3-z_2}}{\sqrt{\left(z_1-z_2\right) \left(z_1-z_3\right)}},b\to \frac{z_1 \sqrt{z_3-z_2}}{\sqrt{\left(z_1-z_2\right) \left(z_1-z_3\right)}},\\
c\to \frac{z_2-z_1}{\sqrt{\left(z_1-z_2\right) \left(z_1-z_3\right)} \sqrt{z_3-z_2}},d\to \frac{\left(z_1-z_2\right) z_3}{\sqrt{\left(z_1-z_2\right) \left(z_1-z_3\right)} \sqrt{z_3-z_2}}
\end{aligned}
\end{equation}
or
\begin{equation}
\begin{aligned}[t]
a\to \frac{\sqrt{z_3-z_2}}{\sqrt{\left(z_1-z_2\right) \left(z_1-z_3\right)}},b\to -\frac{z_1 \sqrt{z_3-z_2}}{\sqrt{\left(z_1-z_2\right) \left(z_1-z_3\right)}},\\
c\to \frac{z_1-z_2}{\sqrt{\left(z_1-z_2\right) \left(z_1-z_3\right)} \sqrt{z_3-z_2}},d\to \frac{\left(z_2-z_1\right) z_3}{\sqrt{\left(z_1-z_2\right) \left(z_1-z_3\right)} \sqrt{z_3-z_2}}.
\end{aligned}
\end{equation}
\end{paragraph}
\end{document}
